\emph{Optimise Investment Returns using a Momentum-Based Portfolio Strategy} was one of the topics under Finsearch'23 conducted by Finance Club, IIT Bombay.

It was a comprehensive, over two month long project, where every group of mentees was provided with ample resources - including books and research papers, and was encouraged to test the practicality of the suggested trading algorithms in context of the Indian Stock markets.

Teams were required to read and understand trading strategies, gather stock price data and run their own backtests to verify multiple established hypothesis.

Since most of the books and articles are written by Wall-Street pundits investigating US stock markets (NYSE), the aim of this project to optimize and implement those strategies on Bombay Stock Exchange (BSE) and National Stock Exchange (NSE) with special focus on the Nifty50 market index.

The main strategy of interest was \textbf{Momentum Trading}, which is one of the simplest \footnote[1]{buy winners} (and yet so complex) trading strategies to understand. We start by studying what momentum is, and how it consistently beats the Efficient Market Hypothesis (EMH) \footnote[2]{EMH suggests that past prices cannot predict future success}.

We then explore the individual and institutional behavior biases which allow momentum strategies to capitalize on market mispricings, and argue that it always generates opportunities for process-driven, long-term focused, disciplined investors. We also discuss how momentum trading is not a competitor to value investing, but a close counterpart instead. \footnote[3]{both rely on same emotional fundamentals}

We then move towards building an effective momentum strategy, and test it in the Indian stock markets. We explore fundamental challenges in momentum trading - finding high quality momentum stocks \footnote[4]{avoid lottery stocks}, seasonality and balancing transactional costs. We discuss methods to calculate and analyze risk by comparing various statistical parameters like $\alpha$ and $\beta$ of our strategy with risk-free options.

One of the unique features of this project was allotment of an experienced mentor for each group, who helped us with both technical and non-technical stuff, and provided supplementary reading material and data sources.

Overall, this was a very hands-on and engaging project and we enjoyed it throughout its course.

