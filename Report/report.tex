\documentclass[a4paper]{article}
\usepackage[margin=0.7in, top=1in, bottom=1in]{geometry}
\usepackage[parfill]{parskip}
\usepackage[utf8]{inputenc}
\usepackage{amsmath,amssymb,amsfonts,amsthm}
\usepackage{fancyhdr}
\usepackage{float}
\usepackage{hyperref}
\usepackage[utf8]{inputenc}
\usepackage[T1]{fontenc}
\usepackage{lmodern}
\usepackage[english]{babel}
\usepackage{tcolorbox}
\tcbuselibrary{theorems}
\newtcbtheorem[number within=section]{defn}{Definition}%
{colback=blue!5,colframe=blue!50!black,fonttitle=\bfseries,before
skip=20pt plus 2pt,after skip=20pt plus 2pt}{dfn}
\newtcbtheorem[number within=section]{thm}{Theorem}%
{colback=red!5,colframe=red!50!black,fonttitle=\bfseries,before
skip=20pt plus 2pt,after skip=20pt plus 2pt}{th}

\pagestyle{fancy}
\fancyhf{}
\chead{Portfolio using Momentum Strategy}
\rhead{Group - D10}
\lhead{Finsearch' 23}
\cfoot{\thepage}


\begin{document}
\pagenumbering{gobble}

\begin{titlepage}
    \vspace*{\fill}
    \begin{center}
        \Large Finsearch'23 - Endterm Report \\[0.5cm]
        \fontsize{40pt}{40pt}\selectfont\textbf{Portfolio using \\ Momentum Strategy} \\[1cm]
        {\Large Anilesh Bansal (22B0914) \\ Satyankar Chandra (22B0967) \\ Soham Dahane (22B0941) \\ Utkarsh Pant (22B0928)\\ Dept of Computer Science, IIT Bombay}\\
    \end{center}
    \vspace*{\fill}
    \begin{center}
        \Large Mentored by Chandrika Aggrawal
    \end{center}
\end{titlepage}

\thispagestyle{empty}

\clearpage

\begin{titlepage}
    \vspace*{\fill}
    \begin{center}
        \includegraphics[width = 10cm]{stonks.jpeg}
    \end{center}
    \vspace{40pt}
    \begin{center}
        \fontsize{25pt}{25pt}\selectfont Buy cheap; buy strong; hold 'em long.
    \end{center}
    \vspace*{\fill}
\end{titlepage}

\tableofcontents

\newpage

\pagenumbering{arabic}
\setcounter{page}{1}

\section{Introduction}
\emph{Optimise Investment Returns using a Momentum-Based Portfolio Strategy} was one of the topics under Finsearch'23 conducted by Finance Club, IIT Bombay.

It was a comprehensive, over two month long project, where every group of mentees was provided with ample resources - including books and research papers, and was encouraged to test the practicality of the suggested trading algorithms in context of the Indian Stock markets.

Teams were required to read and understand trading strategies, gather stock price data and run their own backtests to verify multiple established hypothesis.

Since most of the books and articles are written by Wall-Street pundits investigating US stock markets (NYSE), the aim of this project to optimize and implement those strategies on Bombay Stock Exchange (BSE) and National Stock Exchange (NSE) with special focus on the Nifty50 market index.

The main strategy of interest was \textbf{Momentum Trading}, which is one of the simplest \footnote[1]{buy winners} (and yet so complex) trading strategies to understand. We start by studying what momentum is, and how it consistently beats the Efficient Market Hypothesis (EMH) \footnote[2]{EMH suggests that past prices cannot predict future success}.

We then explore the individual and institutional behavior biases which allow momentum strategies to capitalize on market mispricings, and argue that it always generates opportunities for process-driven, long-term focused, disciplined investors. We also discuss how momentum trading is not a competitor to value investing, but a close counterpart instead. \footnote[3]{both rely on same emotional fundamentals}

We then move towards building an effective momentum strategy, and test it in the Indian stock markets. We explore fundamental challenges in momentum trading - finding high quality momentum stocks \footnote[4]{avoid lottery stocks}, seasonality and balancing transactional costs. We discuss methods to calculate and analyze risk by comparing various statistical parameters like $\alpha$ and $\beta$ of our strategy with risk-free options.

One of the unique features of this project was allotment of an experienced mentor for each group, who helped us with both technical and non-technical stuff, and provided supplementary reading material and data sources.

Overall, this was a very hands-on and engaging project and we enjoyed it throughout its course.


\newpage

\section{Aim and Roadmap}
\subsection*{Project Structure}

The project is subdivided into 3 parts -
\begin{enumerate}
    \item Developing an understanding about momentum trading strategy
    \item Building and testing such strategies in Indian Stock markets
    \item Evaluating the strategy and risk analysis
\end{enumerate}

\bigskip

While building our strategy, these are the ideas we focus on -

\begin{enumerate}
    \item Construction of momentum portfolio (deciding look-back and holding period)
    \item Selecting "smoother" stocks
    \item Considering the seasonality effects in trading
    \item Balancing profit and transactional costs
    \item Reducing risk aversion by diversification
\end{enumerate}

\subsection*{Timeline}

The suggested timeline was as follows -

\begin{itemize}
    \item \textbf{Week 1 - 2:} Read about fundamentals of momentum strategies and understand how they work.
    \item \textbf{Week 3:} Build a trading strategy and evaluate various momentum scenarios.
    \item \textbf{Week 4 - 5:} Backtest strategy on 2010-19 trading data of NSE Nifty50 Companies.
    \item \textbf{Week 6:} Understanding mean revision.
    \item \textbf{Week 7 - 8:} Interpret obtained results and present conclusion as report.
\end{itemize}

At the end, we also had to create a short video explaining our project and presenting our own conclusions.
\newpage

\section{What is Momentum and Why it works}
\input{working.tex}
\newpage

\section{Developing a Momentum Strategy}
\emph{Optimise Investment Returns using a Momentum-Based Portfolio Strategy} was one of the topics under Finsearch'23 conducted by Finance Club, IIT Bombay.

It was a comprehensive, over two month long project, where every group of mentees was provided with ample resources - including books and research papers, and was encouraged to test the practicality of the suggested trading algorithms in context of the Indian Stock markets.

Teams were required to read and understand trading strategies, gather stock price data and run their own backtests to verify multiple established hypothesis.

Since most of the books and articles are written by Wall-Street pundits investigating US stock markets (NYSE), the aim of this project to optimize and implement those strategies on Bombay Stock Exchange (BSE) and National Stock Exchange (NSE) with special focus on the Nifty50 market index.

The main strategy of interest was \textbf{Momentum Trading}, which is one of the simplest \footnote[1]{buy winners} (and yet so complex) trading strategies to understand. We start by studying what momentum is, and how it consistently beats the Efficient Market Hypothesis (EMH) \footnote[2]{EMH suggests that past prices cannot predict future success}.

We then explore the individual and institutional behavior biases which allow momentum strategies to capitalize on market mispricings, and argue that it always generates opportunities for process-driven, long-term focused, disciplined investors. We also discuss how momentum trading is not a competitor to value investing, but a close counterpart instead. \footnote[3]{both rely on same emotional fundamentals}

We then move towards building an effective momentum strategy, and test it in the Indian stock markets. We explore fundamental challenges in momentum trading - finding high quality momentum stocks \footnote[4]{avoid lottery stocks}, seasonality and balancing transactional costs. We discuss methods to calculate and analyze risk by comparing various statistical parameters like $\alpha$ and $\beta$ of our strategy with risk-free options.

One of the unique features of this project was allotment of an experienced mentor for each group, who helped us with both technical and non-technical stuff, and provided supplementary reading material and data sources.

Overall, this was a very hands-on and engaging project and we enjoyed it throughout its course.


\newpage

\section{Strategy considerations}
\emph{Optimise Investment Returns using a Momentum-Based Portfolio Strategy} was one of the topics under Finsearch'23 conducted by Finance Club, IIT Bombay.

It was a comprehensive, over two month long project, where every group of mentees was provided with ample resources - including books and research papers, and was encouraged to test the practicality of the suggested trading algorithms in context of the Indian Stock markets.

Teams were required to read and understand trading strategies, gather stock price data and run their own backtests to verify multiple established hypothesis.

Since most of the books and articles are written by Wall-Street pundits investigating US stock markets (NYSE), the aim of this project to optimize and implement those strategies on Bombay Stock Exchange (BSE) and National Stock Exchange (NSE) with special focus on the Nifty50 market index.

The main strategy of interest was \textbf{Momentum Trading}, which is one of the simplest \footnote[1]{buy winners} (and yet so complex) trading strategies to understand. We start by studying what momentum is, and how it consistently beats the Efficient Market Hypothesis (EMH) \footnote[2]{EMH suggests that past prices cannot predict future success}.

We then explore the individual and institutional behavior biases which allow momentum strategies to capitalize on market mispricings, and argue that it always generates opportunities for process-driven, long-term focused, disciplined investors. We also discuss how momentum trading is not a competitor to value investing, but a close counterpart instead. \footnote[3]{both rely on same emotional fundamentals}

We then move towards building an effective momentum strategy, and test it in the Indian stock markets. We explore fundamental challenges in momentum trading - finding high quality momentum stocks \footnote[4]{avoid lottery stocks}, seasonality and balancing transactional costs. We discuss methods to calculate and analyze risk by comparing various statistical parameters like $\alpha$ and $\beta$ of our strategy with risk-free options.

One of the unique features of this project was allotment of an experienced mentor for each group, who helped us with both technical and non-technical stuff, and provided supplementary reading material and data sources.

Overall, this was a very hands-on and engaging project and we enjoyed it throughout its course.


\newpage

\section{Coding and Backtesting}
\emph{Optimise Investment Returns using a Momentum-Based Portfolio Strategy} was one of the topics under Finsearch'23 conducted by Finance Club, IIT Bombay.

It was a comprehensive, over two month long project, where every group of mentees was provided with ample resources - including books and research papers, and was encouraged to test the practicality of the suggested trading algorithms in context of the Indian Stock markets.

Teams were required to read and understand trading strategies, gather stock price data and run their own backtests to verify multiple established hypothesis.

Since most of the books and articles are written by Wall-Street pundits investigating US stock markets (NYSE), the aim of this project to optimize and implement those strategies on Bombay Stock Exchange (BSE) and National Stock Exchange (NSE) with special focus on the Nifty50 market index.

The main strategy of interest was \textbf{Momentum Trading}, which is one of the simplest \footnote[1]{buy winners} (and yet so complex) trading strategies to understand. We start by studying what momentum is, and how it consistently beats the Efficient Market Hypothesis (EMH) \footnote[2]{EMH suggests that past prices cannot predict future success}.

We then explore the individual and institutional behavior biases which allow momentum strategies to capitalize on market mispricings, and argue that it always generates opportunities for process-driven, long-term focused, disciplined investors. We also discuss how momentum trading is not a competitor to value investing, but a close counterpart instead. \footnote[3]{both rely on same emotional fundamentals}

We then move towards building an effective momentum strategy, and test it in the Indian stock markets. We explore fundamental challenges in momentum trading - finding high quality momentum stocks \footnote[4]{avoid lottery stocks}, seasonality and balancing transactional costs. We discuss methods to calculate and analyze risk by comparing various statistical parameters like $\alpha$ and $\beta$ of our strategy with risk-free options.

One of the unique features of this project was allotment of an experienced mentor for each group, who helped us with both technical and non-technical stuff, and provided supplementary reading material and data sources.

Overall, this was a very hands-on and engaging project and we enjoyed it throughout its course.


\newpage

\section{Results Discussion}
\emph{Optimise Investment Returns using a Momentum-Based Portfolio Strategy} was one of the topics under Finsearch'23 conducted by Finance Club, IIT Bombay.

It was a comprehensive, over two month long project, where every group of mentees was provided with ample resources - including books and research papers, and was encouraged to test the practicality of the suggested trading algorithms in context of the Indian Stock markets.

Teams were required to read and understand trading strategies, gather stock price data and run their own backtests to verify multiple established hypothesis.

Since most of the books and articles are written by Wall-Street pundits investigating US stock markets (NYSE), the aim of this project to optimize and implement those strategies on Bombay Stock Exchange (BSE) and National Stock Exchange (NSE) with special focus on the Nifty50 market index.

The main strategy of interest was \textbf{Momentum Trading}, which is one of the simplest \footnote[1]{buy winners} (and yet so complex) trading strategies to understand. We start by studying what momentum is, and how it consistently beats the Efficient Market Hypothesis (EMH) \footnote[2]{EMH suggests that past prices cannot predict future success}.

We then explore the individual and institutional behavior biases which allow momentum strategies to capitalize on market mispricings, and argue that it always generates opportunities for process-driven, long-term focused, disciplined investors. We also discuss how momentum trading is not a competitor to value investing, but a close counterpart instead. \footnote[3]{both rely on same emotional fundamentals}

We then move towards building an effective momentum strategy, and test it in the Indian stock markets. We explore fundamental challenges in momentum trading - finding high quality momentum stocks \footnote[4]{avoid lottery stocks}, seasonality and balancing transactional costs. We discuss methods to calculate and analyze risk by comparing various statistical parameters like $\alpha$ and $\beta$ of our strategy with risk-free options.

One of the unique features of this project was allotment of an experienced mentor for each group, who helped us with both technical and non-technical stuff, and provided supplementary reading material and data sources.

Overall, this was a very hands-on and engaging project and we enjoyed it throughout its course.


\newpage

\section{Risk Analysis and Management}
\emph{Optimise Investment Returns using a Momentum-Based Portfolio Strategy} was one of the topics under Finsearch'23 conducted by Finance Club, IIT Bombay.

It was a comprehensive, over two month long project, where every group of mentees was provided with ample resources - including books and research papers, and was encouraged to test the practicality of the suggested trading algorithms in context of the Indian Stock markets.

Teams were required to read and understand trading strategies, gather stock price data and run their own backtests to verify multiple established hypothesis.

Since most of the books and articles are written by Wall-Street pundits investigating US stock markets (NYSE), the aim of this project to optimize and implement those strategies on Bombay Stock Exchange (BSE) and National Stock Exchange (NSE) with special focus on the Nifty50 market index.

The main strategy of interest was \textbf{Momentum Trading}, which is one of the simplest \footnote[1]{buy winners} (and yet so complex) trading strategies to understand. We start by studying what momentum is, and how it consistently beats the Efficient Market Hypothesis (EMH) \footnote[2]{EMH suggests that past prices cannot predict future success}.

We then explore the individual and institutional behavior biases which allow momentum strategies to capitalize on market mispricings, and argue that it always generates opportunities for process-driven, long-term focused, disciplined investors. We also discuss how momentum trading is not a competitor to value investing, but a close counterpart instead. \footnote[3]{both rely on same emotional fundamentals}

We then move towards building an effective momentum strategy, and test it in the Indian stock markets. We explore fundamental challenges in momentum trading - finding high quality momentum stocks \footnote[4]{avoid lottery stocks}, seasonality and balancing transactional costs. We discuss methods to calculate and analyze risk by comparing various statistical parameters like $\alpha$ and $\beta$ of our strategy with risk-free options.

One of the unique features of this project was allotment of an experienced mentor for each group, who helped us with both technical and non-technical stuff, and provided supplementary reading material and data sources.

Overall, this was a very hands-on and engaging project and we enjoyed it throughout its course.


\newpage

\section{Mean Revision}
\emph{Optimise Investment Returns using a Momentum-Based Portfolio Strategy} was one of the topics under Finsearch'23 conducted by Finance Club, IIT Bombay.

It was a comprehensive, over two month long project, where every group of mentees was provided with ample resources - including books and research papers, and was encouraged to test the practicality of the suggested trading algorithms in context of the Indian Stock markets.

Teams were required to read and understand trading strategies, gather stock price data and run their own backtests to verify multiple established hypothesis.

Since most of the books and articles are written by Wall-Street pundits investigating US stock markets (NYSE), the aim of this project to optimize and implement those strategies on Bombay Stock Exchange (BSE) and National Stock Exchange (NSE) with special focus on the Nifty50 market index.

The main strategy of interest was \textbf{Momentum Trading}, which is one of the simplest \footnote[1]{buy winners} (and yet so complex) trading strategies to understand. We start by studying what momentum is, and how it consistently beats the Efficient Market Hypothesis (EMH) \footnote[2]{EMH suggests that past prices cannot predict future success}.

We then explore the individual and institutional behavior biases which allow momentum strategies to capitalize on market mispricings, and argue that it always generates opportunities for process-driven, long-term focused, disciplined investors. We also discuss how momentum trading is not a competitor to value investing, but a close counterpart instead. \footnote[3]{both rely on same emotional fundamentals}

We then move towards building an effective momentum strategy, and test it in the Indian stock markets. We explore fundamental challenges in momentum trading - finding high quality momentum stocks \footnote[4]{avoid lottery stocks}, seasonality and balancing transactional costs. We discuss methods to calculate and analyze risk by comparing various statistical parameters like $\alpha$ and $\beta$ of our strategy with risk-free options.

One of the unique features of this project was allotment of an experienced mentor for each group, who helped us with both technical and non-technical stuff, and provided supplementary reading material and data sources.

Overall, this was a very hands-on and engaging project and we enjoyed it throughout its course.


\newpage

\section{Conclusion}
\emph{Optimise Investment Returns using a Momentum-Based Portfolio Strategy} was one of the topics under Finsearch'23 conducted by Finance Club, IIT Bombay.

It was a comprehensive, over two month long project, where every group of mentees was provided with ample resources - including books and research papers, and was encouraged to test the practicality of the suggested trading algorithms in context of the Indian Stock markets.

Teams were required to read and understand trading strategies, gather stock price data and run their own backtests to verify multiple established hypothesis.

Since most of the books and articles are written by Wall-Street pundits investigating US stock markets (NYSE), the aim of this project to optimize and implement those strategies on Bombay Stock Exchange (BSE) and National Stock Exchange (NSE) with special focus on the Nifty50 market index.

The main strategy of interest was \textbf{Momentum Trading}, which is one of the simplest \footnote[1]{buy winners} (and yet so complex) trading strategies to understand. We start by studying what momentum is, and how it consistently beats the Efficient Market Hypothesis (EMH) \footnote[2]{EMH suggests that past prices cannot predict future success}.

We then explore the individual and institutional behavior biases which allow momentum strategies to capitalize on market mispricings, and argue that it always generates opportunities for process-driven, long-term focused, disciplined investors. We also discuss how momentum trading is not a competitor to value investing, but a close counterpart instead. \footnote[3]{both rely on same emotional fundamentals}

We then move towards building an effective momentum strategy, and test it in the Indian stock markets. We explore fundamental challenges in momentum trading - finding high quality momentum stocks \footnote[4]{avoid lottery stocks}, seasonality and balancing transactional costs. We discuss methods to calculate and analyze risk by comparing various statistical parameters like $\alpha$ and $\beta$ of our strategy with risk-free options.

One of the unique features of this project was allotment of an experienced mentor for each group, who helped us with both technical and non-technical stuff, and provided supplementary reading material and data sources.

Overall, this was a very hands-on and engaging project and we enjoyed it throughout its course.


\newpage

\appendix
\section{Data Collection}
\emph{Optimise Investment Returns using a Momentum-Based Portfolio Strategy} was one of the topics under Finsearch'23 conducted by Finance Club, IIT Bombay.

It was a comprehensive, over two month long project, where every group of mentees was provided with ample resources - including books and research papers, and was encouraged to test the practicality of the suggested trading algorithms in context of the Indian Stock markets.

Teams were required to read and understand trading strategies, gather stock price data and run their own backtests to verify multiple established hypothesis.

Since most of the books and articles are written by Wall-Street pundits investigating US stock markets (NYSE), the aim of this project to optimize and implement those strategies on Bombay Stock Exchange (BSE) and National Stock Exchange (NSE) with special focus on the Nifty50 market index.

The main strategy of interest was \textbf{Momentum Trading}, which is one of the simplest \footnote[1]{buy winners} (and yet so complex) trading strategies to understand. We start by studying what momentum is, and how it consistently beats the Efficient Market Hypothesis (EMH) \footnote[2]{EMH suggests that past prices cannot predict future success}.

We then explore the individual and institutional behavior biases which allow momentum strategies to capitalize on market mispricings, and argue that it always generates opportunities for process-driven, long-term focused, disciplined investors. We also discuss how momentum trading is not a competitor to value investing, but a close counterpart instead. \footnote[3]{both rely on same emotional fundamentals}

We then move towards building an effective momentum strategy, and test it in the Indian stock markets. We explore fundamental challenges in momentum trading - finding high quality momentum stocks \footnote[4]{avoid lottery stocks}, seasonality and balancing transactional costs. We discuss methods to calculate and analyze risk by comparing various statistical parameters like $\alpha$ and $\beta$ of our strategy with risk-free options.

One of the unique features of this project was allotment of an experienced mentor for each group, who helped us with both technical and non-technical stuff, and provided supplementary reading material and data sources.

Overall, this was a very hands-on and engaging project and we enjoyed it throughout its course.


\newpage

\section{Credits}
\emph{Optimise Investment Returns using a Momentum-Based Portfolio Strategy} was one of the topics under Finsearch'23 conducted by Finance Club, IIT Bombay.

It was a comprehensive, over two month long project, where every group of mentees was provided with ample resources - including books and research papers, and was encouraged to test the practicality of the suggested trading algorithms in context of the Indian Stock markets.

Teams were required to read and understand trading strategies, gather stock price data and run their own backtests to verify multiple established hypothesis.

Since most of the books and articles are written by Wall-Street pundits investigating US stock markets (NYSE), the aim of this project to optimize and implement those strategies on Bombay Stock Exchange (BSE) and National Stock Exchange (NSE) with special focus on the Nifty50 market index.

The main strategy of interest was \textbf{Momentum Trading}, which is one of the simplest \footnote[1]{buy winners} (and yet so complex) trading strategies to understand. We start by studying what momentum is, and how it consistently beats the Efficient Market Hypothesis (EMH) \footnote[2]{EMH suggests that past prices cannot predict future success}.

We then explore the individual and institutional behavior biases which allow momentum strategies to capitalize on market mispricings, and argue that it always generates opportunities for process-driven, long-term focused, disciplined investors. We also discuss how momentum trading is not a competitor to value investing, but a close counterpart instead. \footnote[3]{both rely on same emotional fundamentals}

We then move towards building an effective momentum strategy, and test it in the Indian stock markets. We explore fundamental challenges in momentum trading - finding high quality momentum stocks \footnote[4]{avoid lottery stocks}, seasonality and balancing transactional costs. We discuss methods to calculate and analyze risk by comparing various statistical parameters like $\alpha$ and $\beta$ of our strategy with risk-free options.

One of the unique features of this project was allotment of an experienced mentor for each group, who helped us with both technical and non-technical stuff, and provided supplementary reading material and data sources.

Overall, this was a very hands-on and engaging project and we enjoyed it throughout its course.


\newpage




% \begin{figure}[htb]\hspace*{\fill}
% \begin{game}{2}{2}[Player 1][Player 2][TCP Playoff Game]
% & L & R\\
% T &2,2 &100,35\\
% B &3,0 &350,16
% \end{game}
% \hspace*{\fill}\label{tcp}
% \end{figure}




% \begin{defn}{This is my title}{theoexample}
%   This is the text of the theorem. The counter is automatically assigned and,
%   in this example, prefixed with the section number. This theorem is numbered with
%   \ref{th:theoexample} and is given on page \pageref{th:theoexample}.
% \end{defn}


\end{document}
